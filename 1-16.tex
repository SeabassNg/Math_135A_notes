\section*{1/16}
  \underline{Example}: 3 married couples. Take seats around a table at random.
  we want to figure out $P(\text{no wife sits next to their husband})$.\\
  First compute the complement, but unions are hard compute, so we can compute
  this instead:
  $1 - (P(A_1) + P(A_2) + P(A_3) - P(A_1 \bigcap A_2) - P(A_1 \bigcap A_3) - 
  P(A_2 \bigcap A_3) + P(A_1 \bigcap A_2 \bigcap A_3)$ where $A_i = \{ 
  \text{wife$_i$ and husband$_i$ sit together}\}$\\
  $P(A_1) = \frac{2}{5} = P(A_2) = P(A_3)$\\
  $P(A_1 \bigcap A_2) = P(A_1 \bigcap A_3) = P(A_2 \bigcap A_3) = \frac{2 * 2 * 3!}{5!} = \frac{1}{5}$.\\
  Fix husband 1. 2 places for wife to be. 3! ways of picking where second couple
  sits. 2 ways to vary the couples.
  $P(A_1 \bigcap A_2 \bigcap A_3) = \frac{2 * 2! * 2^2}{5!}$\\
  Our answer is $\frac{4}{15}$\\\\
  \underline{Example}: Birthday Problem\\
  $n$ possible birthdays, sample $k$ times with replacement.\\
  $P(\text{at least 2 people share a birthday}) = 1 - 
  P(\text{no shared birthdays}) = 1 - \frac{n*(n-1)\ldots (n-k+1)}{n^k}$\\
  \underline{Example(Massachusetts lottery)}: $n = 10,000$, $k = 660$\\
  Pick a 4 digit number a day. 660 days have passed with no numbers being
  the same. What is this probability?
  $P(\text{no shared birthdays or lottery numbers}) = 2.191 * 10^{-10}$\\\\
  \underline{Example}: $P(\text{all $n$ birthday are represented})$?\\
  Suppose all birthdays are represented. That probability is $\frac{n!}{n^n}$.\\
  What about our problem? The probability is then $1 - 
  P(\bigcup_{i=1}^n A_i)$ where $A_i = \{\text{$i$th birthday is missing}\}$
  $P(A_1) = \frac{(n-1)^k}{n^k} = P(A_i)$ $\forall i$\\
  $P(A_1 \bigcap A_2) = \frac{(n-2)^k}{n^k} = P(A_i \bigcap A_j)$ where $i < 
  j$\\
  $\vdots$\\
  We then get $1 - n\left(\frac{n-1}{n}\right)^k + \binom{n}{2}\left(
  \frac{n-2}{n}\right)^k - \ldots = \sum_{i = 0}^{n-1} \binom{n}{i} (-1)^i
  (1 - \frac{i}{n})^k$\\
  As it turns out for the above formula, the probability is $\frac{n!}{n^n}$
  for $k = n$, but about 0 when $k < n$. This is similar to the sterling 
  numbers of the second kind.\\\\
  \underline{Example}: Matching Problems\\
    $n$ employees, each buys a gift. These are then assigned at random.\\
    \begin{eqnarray*}
      P(\text{somebody gets own gift}) &=& P(\bigcup_{i=1}^{n}A_i) \text{ where } A_i = \{\text{$i$th person gets own gift}\}\\
      P(A_1) & = & \frac{1}{n}\\
      & = & P(A_i)\\
      P(A_1 \bigcap A_2) & = & \frac{(n-2)!}{n!} \\
      & = & \frac{1}{n(n-1)}\\
      P(\ldots) & = & n * \frac{1}{n} - \binom{n}{2}\frac{1}{n(n-1)} + \ldots \\
       & = & \text{ (as $n \to \infty$) } 1 - \frac{1}{e}
    \end{eqnarray*}
