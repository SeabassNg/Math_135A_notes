\section*{3/2}
  \subsection*{Conditional distributions}
    Discrete cases:
    $$
      p_{x|y}(x|y) = P(X=x | Y=y) = \frac{P(X=x,Y=y)}{P(Y=y)}
    $$
    Continuous cases:
    $$
      f_{x|y}(x|y) = \frac{f(x,y)}{f_Y(y)}
    $$
    The latter is a trivial formula since numerator and denominator is always
    0 if the denominator is $0$.
    It should be:
    \begin{eqnarray*}
      f_{x|y}(x|y) &=& \frac{P(X = x+dx, Y=y+dy)}{P(Y=y + dy)}\\
        & = & \frac{f(x,y)\,dx\,dy}{f_Y(y)dy)}\\
        & = & \frac{f(x,y)}{f_Y(y)}dx\\
        & \approx & P(X = x + dx| Y = y + dy)
    \end{eqnarray*}
    \underline{Example}: Let $(x,y)$ be a random point on the isosceles triangle
      of sides, 1.\\
      $f(x,y) = 2$ on the triangle.\\
      Compute $f_{x|y}(x|y)$.\\
      Let's say I give you $y$, then you know that $x$ is between 0 and $1-y$.\\
      \begin{eqnarray*}
        f_Y(y) & = & \int_{0} ^{1-y} 2\,dx\\
          & = & 2(1-y)
      \end{eqnarray*}
      where $y \in [0,1]$\\
      $$
        f_{x|y}(x|y) = \begin{cases}\frac{2}{2(1-y)} & 0 \le x \le 1 - y\\
        0 & \text{otherwise}\end{cases}
      $$
    \underline{Example}: Suppose
    $$
      f(x,y) = \begin{cases} \frac{21}{4}x^2y & x^2 \le y \le 1\\ 
      0 &\text{otherwise}\end{cases}
    $$
    Compute $f_{x|y}(x,y)$\\
    $$
      f_Y(y) = \frac{21}{4} y\int_{-\sqrt{y}}^{\sqrt{y}} x^2\,dx = \frac{7}{2}
      y^{\frac{5}{2}}
    $$
    where $y \in [0,1]$\\
    $$
      f_{x|y}(x|y) = \frac{\frac{21}{4}x^2y}{\frac{7}{2}y^{5/2}} = \frac{3}{2}
        x^2y^{-3/2}
    $$
    where $-\sqrt{y} \le x \le \sqrt{y}$\\\\
    How about $P(X \ge Y | Y = y)$? Normally, this makes no sense because
    we're given something that has probability, 0. This is interpreted as
    $$
      \int_y^{\sqrt{y}} \frac{3}{2}x^2y^{-3/2}\,dx = \int_y^{\sqrt{y}} f_{X|Y}
      (x|y)\,dx
    $$\\\\
  $$
    EX = \begin{cases} \sum_i x_i p(x_i) & \text{where $X$ is a discerete with probability, $p$}.\\ \int_{-\infty}^{\infty}xf(x)\,dx & \text{where $X$ is continuous with density, $f$}\end{cases}\\
  $$
  $$
    Eg(X) = \begin{cases} \sum_i g(x_i) p(x_i) & \text{where $X$ is a discerete with probability, $p$}.\\ \int_{-\infty}^{\infty} g(x)f(x)\,dx & \text{where $X$ is continuous with density, $f$}\end{cases}
  $$
  If we have $(X,Y)$, then
  $$
    Eg(X) = \begin{cases} \sum_{x,y} g(x,y) P(X=x, Y=y) & \text{where $X$ is a discerete with probability, $p$}.\\ \int_{-\infty}^{\infty}\int_{-\infty}^{\infty} g(x,y)f(x,y)\,dx\,dy & \text{where $X$ is continuous with density, $f$}\end{cases}
  $$
  \underline{Example}: Roll two dice, with $X$ being the number on the 1st die
    and $Y$ be the number on the 2nd die.\\
    Let $Z$ be the difference between $X$ and $Y$.\\
    Compute $P(Z = i)$ where $i = 0, 1, 2, 3, 4, 5$\\
    Then, $$
      E|X - y| = \sum_{i = 1}^6 \sum_{i = 1}^6 |i - j| \frac{1}{36}
    $$\\
    Alternatively,
    $$
      EZ = \sum_{k = 0}^5 kP(Z = k)
    $$
    \underline{Example}: Choose a random point, $(x,y)$ on the right half of 
    the unit circle. Compute EX.\\

