\section*{1/28}
  \underline{Example}: We have a fair coin and an unfair coin which always 
    tosses head. Choose one at random, toss it twice. It comes out heads.\\
    $P(\text{the coin is fair})$?\\
    Let $F = \{\text{fair}\}$, $U = \{\text{unfair}\}$, and $E = \{
    \text{both tosses H\\}$.\\
    Using the First Bayes' formula, you can compute $E$ a-priori.\\
    $P(F) = P(U) = \frac{1}{2}$. You have equal probability of picking
    fair and unfair coin a-priori.\\
    $P(E|F) = \frac{1}{4}$. Given you picked fair coin, probability of both 
    tosses heads\\
    $P(E|U) = 1$. Given you picked unfair coin, probability of both tosses
    tails.\\
    Our probability is $\frac{\frac{1}{2}\frac{1}{4}}{\frac{1}{2}\times
    \frac{1}{4} + \frac{1}{2} \times 1} = \frac{1}{5}$.\\\\
  \underline{Example}: OJ Simpson\\
    A Dershowitz: not probable because $P(\text{an wife abuser kills wife}) =
    .001$\\
    JF Merz and JP Caulkins cancels his logic because Dershowiz did not take 
    into account the fact that someone actually murdered someone.\\
    Let $S = \{\text{murdered women}\}$ with size 4936 (in 1992), out of
    which 1430 were killed by partners.\\
    Let $A = \{\text{partner abused}\}$\\
    Let $M = \{\text{partner murdered}\}$\\
    $P(M) = .29$\\
    $P(M^c) = .71$\\
    You're interested in $P(M|A)$.\\
    $P(A | M) = .5$\\
    $P(A | M^c) = .05$\\
    $P(M|A) = \frac{P(M)P(A |M)}{P(M)P(A|M)+ P(M^c)P(A| M^c)} \approx 80\%$\\\\
  \subsection*{Independence}
    Events, $E$ and $F$ are independent if
    \begin{itemize}
      \item $P(E|F) = P(E)$
      \item $P(E \bigcup F) = P(E)P(F)$
    \end{itemize}
    \underline{Note}: If $E$ and $F$ are independent, so are $E$ and $F^c$. 
      Then,
      \begin{itemize}
        \item $P(E \bigcup F) = P(E)P(F^c)$
        \item Then, $P(E) - P(E \bigcap F) = P(E) - P(E)P(F)$
      \end{itemize}
    \underline{Note}: Independence is often an assumption rather than something
    we find.\\\\
    \underline{Example}: Pull a random card from a full deck.\\
    Let $E = \{ \text{red} \}$\\
    Let $F = \{ \text{ace} \}$\\
    $P(E) = \frac{1}{2}$, $P(E) = \frac{1}{13}$, and $P(E \bigcap F) = 
    \frac{2}{52}$, so they're independent.\\\\
    Pull two cards out of the deck sequentially.\\
    $E = \{\text{first card is ace}\}$\\
    $F = \{\text{2nd card is ace}\}$\\
    $P(E) = P(F) = \frac{1}{13}$\\
    $P(F | E) = \frac{3}{51}$, so not independent.\\\\
    \underline{Example}: Toss 2 fair coins\\
    $E = \{ \text{ head on 1st try}\}$\\
    $F = \{ \text{ head on 2nd try}\}$\\
    This is independent.\\\\
    Independence of $E_1, E_2, \ldots, E_n$\\
    $P(E_{i_1} \bigcap \ldots \bigcap E_{i_k}) = P(E_{i_1})P(E_{i_2})\ldots 
    P(E_{i_k})$ where $1 \le i_1 \le i_2 \le \ldots \le i_k \le n$.\\\\
    \underline{Example}:
      Roll a four sided fair die.\\
      $A = \{ 1, 2\}$, $B = \{2, 3\}$, $C = \{1, 4\}$\\
      $P(A) = P(B) = P(C) = \frac{1}{2}$\\
      $P(A \bigcap B) = P(A \bigcap C) = P(B \bigcap C)$.\\
      These events are pairwise independent = $\frac{1}{4}$\\
      Are all 3 independent?\\
      $P(A \bigcap B \bigcap C) = \frac{1}{4} \not= \frac{1}{8}$\\
      $A \bigcap B \subset C$\\
      $A$ and $B$ cannot be independent if they are disjoint. If one happens,
      the other cannot happen. This is not independent because that would
      mean that it can still happen with the same probability.
