\section*{2/11}

\underline{Example}: {\bf Poisson distribution and law.} 
\vskip0.2cm 

Assume a crime has been committed. It is known that the 
perpetrator has certain characteristics, which occur
with small frequency $p$ (say, $10^{-8}$) in the population of size $n$
(say, $10^8$). 
A person who matches these characteristics has been 
found at random (e.g., at a routine traffic stop,  or by
airport security) and, since $p$ is so small, charged with the 
crime. There is no other evidence. What should the defense be? 

Let's start with a mathematical model of this situation. Assume 
that $N$ is the number of people with given characteristics. This 
is a Binomial random variable, but given the assumption we can 
easily assume it's Poisson with $\lambda=np$. Choose a random person from 
among these $N$, call that person $C$, the criminal. Then, independently, 
choose at random another person, $A$, who is arrested. The question 
is whether $C=A$, that is, whether the arrested person is guilty. Mathematically, 
we can formulate the problem like this: 
%$$
%P(C=A\,|\, N\ge 1)  =  \frac{P(C=A, N\ge 1)}{P(N\ge 1)}. 
%$$
We need to condition as the experiment cannot even be performed when $N=0$. 
Now by the first Bayes formula, 
\begin{eqnarray*}
P(C=A, N\ge 1)&= &\sum_{k=1}^{\infty} P(C=A, N\ge 1\,|\,N=k)\cdot P(N=k)\\
&= &\sum_{k=1}^\infty P(C=A\,|\,N=k)\cdot P(N=k)\\
\end{eqnarray*}
and 
$$
P(C=A\,|\,N=k)=\frac{1}{k}, 
$$
so 
$$
P(C=A, N \ge 1)=\sum_{k=1}^{\infty} \frac{1}{k} \cdot \frac{\lambda^k}{k!}\cdot 
e^{-\lambda}. 
$$
The answer to the question then is 
$$
P(C=A\,|\, N\ge 1)=\frac{e^{-\lambda}}{1-e^{-\lambda}}\cdots
\sum_{k=1}^{\infty} \frac{\lambda^k}{k\cdot k!}. 
$$
There is no closed-form expression for the sum, but it can be easily
computed numerically. The defense may claim 
that the probability of innocence, $1-$(the above probability), 
is about $0.2330$ when $\lambda=1$, quite enough for a reasonable
doubt. 

This model was in fact tested in court, in the famous {\it 
People v.~Collins\/} case, a 1968 jury trial in Los Angeles. In this instance,  
it was claimed by prosecution (on flimsy grounds) that 
$p=1/12,000,000$ and $n$ would be the adult couples in the LA area, 
say $n=5,000,000$. The jury convicted the charged couple in a robbery 
on the basis of the prosecutor's claim that, due to low $p$, ``the chances of there 
being another couple [with specified characteristics, in the LA area]
must be one in a billion.'' The Supreme Court of California reversed 
the conviction, and gave two reasons. The first reason was insufficient foundation 
for estimate for $p$. The second reason was that 
$$
P(N\ge 2\,|\, N\ge 1)=\frac{1-e^{-\lambda}-\lambda e^{-\lambda}}{1-e^{-\lambda}}
$$ 
from the prosecutor's claim is in fact much larger than he claimed,
namely about $0.1939$. This
is about twice as big as the probability of innocence, which 
by the above formula would be about $0.1015$. 
  \subsection*{Geometric random variables}
    $X =$ number trials required for the first success in independent trials
    with success probability $p$.\\
    $P(X = n) = p(1-p)^{n-1}$ with $n = 1, 2, \ldots$\\
    $EX = \frac{1}{p}$\\
    $Var(X) = \frac{1-p}{p^2}$\\
    $P(X \ge n) = \sum_{n}^{\infty} p(1-p)^{n-1} = (1-p)^{n-1}$\\
    $P(X \ge n + k | X \ge k) = \frac{(1-p)^{n+k-1}}{(1-p)^{k}}$\\\\
    \underline{Example}: $X = $ number of tosses required for 1st heads (in
    a fair coin).\\
      $EX = 2$.\\\\
    \underline{Example}: you roll a die, your opponent tosses a coin.\\
    Roll 6 $\to$ you win.\\
    Now roll 6, opponent tosses Heads $\to$ you lose\\
    Otherwise, the game repeats.\\
    On the average, how many steps does the game last?\\
    $P(\text{game decided on step 1}) = \frac{1}{6} + \frac{5}{6}\frac{1}{2}$\\
    $EN = \frac{1}{\frac{1}{6} + \frac{5}{6}\frac{1}{2}} = \frac{12}{7}$

  \subsection*{Continuous random variables}
    Let $X$ be a continuous random variable.\\
    If $P(X \in B) = \int f(x)\,dx$ where $f$ is a nonnegative function which
    $\int_{-\infty}^{\infty} f(x)\,dx = 1$\\
    The function $f = f_X$ is called the density of $X$.\\
    $P(X \in [a,b]) = P(a \le X \le b)$\\
    $P(X = a) = 0$\\
    $P(X \le a) = \int_{-\infty}^a f(x)\,dx = P(X < a)$.\\

  \subsection*{distribution functions}
    $F(x) = P(X \le x) = \int_{-\infty}^x f(s)\,ds$ where $F$ is the 
    distribution function.\\
    $\frac{dF}{dx} = f$\\
    $EX = \int_{-\infty}^{\infty}f(x)\,dx$\\\\
    \underline{Example}: $f(x) = \begin{cases} cx & 0 < x < 4\\ 0 & 
      \text{otherwise} \end{cases}$\\
    \begin{enumerate}
      \item Determine $c$
      \item $P(1 \le X \le 2)$
      \item $EX, EX^2$
    \end{enumerate}
    Determine $c$.\\
    $\int_0^4 cx\,dx = 1$, so $c = \frac{1}{8}$\\
    $\int_1^2 \frac{x}{8}\,dx = \frac{3}{16}$\\
    $EX = \int_0^4 \frac{x^2}{8}\,dx = \frac{1}{3}$\\
