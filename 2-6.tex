\section*{2/6}
  \begin{definition} $X$ is a discrete random variable with possible
  values, $x_i$. Then, the expected values (average, mean) of $X$ is
  $EX = \sum_i x_i P(x = x_i) = \sum_i x_i p(x_i)$\\
  For any function, $g$, $Eg(X) = \sum_i g(x_i) P(X = x_i)$
  \end{definition}
  \noindent\underline{Example}: $X$, a random variable with $P(X = 1) = .2, P(X=2)
    = .3, and P(X=3) = .5$\\
    What should be the average value of $X$?\\
    In a population of $n$ independent realization of $X$, about $.2n$ will
    have $X = 1$, $.3n$ will have $X = 2$, $.5n$ will have $X = 3$.\\\\
    Average: $\frac{1 \times .2n + 2 \times .3n + 3 \times .5n}{n} = 2.3$\\
    The expected value is 2.3.\\\\
    $X$ is a discrete random variable.\\
    $\mu = EX$ where $\mu$ is the averag where $\mu$ is the average.\\\\
    How about deviations? One way to find it, $E|X - \mu|$. Problem is that
    absolute values are annoying.\\
    Therefore, we devise, $E(x - \mu)^2$. This is known as the Variance of X.\\
    Let the standard devision, $SD(X) = \sigma(X) = \sqrt{Var(X)} = 
    \sqrt{E(x-\mu)^2}$\\
    \begin{eqnarray*}
      Var(X) & = & E[(X - \mu)^2]\\
        & = & E[X^2 - 2\mu X +\mu^2]\\
        & = & E(X^2) - 2\mu E(X) + \mu^2\\
        & = & E(X^2) - \mu^2 = E(X^2) - (EX)^2\\
    \end{eqnarray*}
  \underline{Example}: Previous example, continued\\
    $E(X^2) = 1^2 \times .2 + 2^2 \times .3 + 3^2 \times .5 = 6.7$\\
    $Var(X) = 6.7 - (2.3)^2 = 1.49$\\
    $\sigma(X) = \sqrt{Var(X)} = 1.19$\\\\
  \underline{Example}: $X$ = number shown on a die\\
    $E(X) = \frac{1}{6}(1 + 2 + \ldots + 6 = \frac{7}{2} = 3.5$\\
    $EX^2 = \frac{1}{6}(1 + 2^2 + \ldots + 6^2) = \frac{91}{6} = 15.16$\\\\
  \subsection*{Probability Mass functions}
    \begin{enumerate}
      \item Uniform
      \begin{enumerate}
        \item $P(X = X_1) = \frac{1}{n}$, where $i = 1, \ldots, n$. 
        \item $EX = \frac{x_1 + \ldots x_n}{n}$
      \end{enumerate}
      \item Bernoulli: An indicator random variable. Assume $A$ is an event with
        probability $p$. 
      \begin{enumerate}
        \item \underline{Indicator of A}: $I_A - X_A = 1_A =  
          \begin{cases} 1 & \text{ if satisfied}\\ 0 & \text{ otherwise}
          \end{cases}$. 
        \item $E(I_A) = p$
      \end{enumerate}
      \item Binomial random variable: Number of successes in $n$ independent
        trials, each of which is a success with probability $p$.
      \begin{enumerate}
        \item $X$ is $Binomial(n,p)$, so $P(X=i) = \binom{n}{i}p^i(1-p)^{n-i}$
          where $i = 0, \ldots, n$.
        \item $EX = np$
        \item $Var(X) = np(1-p)$
      \end{enumerate}
      \item Poisson random variable. $X = Poisson(\lambda)$
      \begin{enumerate}
        \item $P(X = k) = \frac{\lambda^k}{k!} e^{-\lambda}$ where $k = 0, 1, 2,
        \ldots$
      \end{enumerate}
    \end{enumerate}
    \underline{Example}: Let $X$ = number of heads in 50 tosses of a fair 
    coin.\\
    $X$ is $Binomial(50, \frac{1}{2})$\\
    $P(X \le 10) = \sum_{i = 0}^{10} \binom{10}{1} \frac{1}{2^{50}}$\\\\
    \underline{Example}: Let $d$ = dominant genes and $r$ = recessive gene.\\
      $dd$ is called the pure dominant gene, $dr$ is called hybrid, and $rr$
      puure recessive recessive". $dd$ and $dr$ produces the quality of
      dominant genes. $rr$ procues the quality of recessive genes.\\\\
      Assume that both parents are hybrid and have $n$ children, what is $P(
      \text{at least two will be $rr$})$?\\
      Each child, independent, gets one of the genes at random from each 
      parent.\\
      $P(\text{pure recessive child}) = \frac{1}{4}$\\
      $X$ is the number of $rr$ children.\\
      $X = Binomial(n, \frac{1}{4})$\\
      $P(X \ge 2) = 1 - P(X = 0) - P(X = 1) = 1 - (\frac{3}{4})^n - n\frac{1}{4}
      (\frac{3}{4})^{n-1}$\\
