\section*{3/6}
  \subsection*{Expectation}
    $$
      Eg(X,Y) = \int\int g(x,y)f(x,y)\,dxdy \text{(Continuous case)}
    $$
    $$
      Eg(X,Y) = \sum_{x,y}g(x,y)p(x,y) \text{(discrete case)}
    $$
    \underline{Example}: Hw8.6\\
      Joint mass function with (product of $i,j$
      \begin{tabular}{c | c | c| c | c}
        $i \backslash j$  & 1 & 2 & 3 & 4\\
        \hline
        1 &  .1 (1) & .1 (2) & .1 (3) & .1 (4)\\
        \hline
        2 &  .1 (2) & .1 (4) & .1 (6) & .1 (8)\\
        \hline
        3 &  .1 (3) & .1 (6) & 0 & 0\\
        \hline
        4 &  .1 (4) & 0 & 0 & 0\\
      \end{tabular}

    \noindent\underline{Example}: $(X,Y)$ is a random point in the right 
    triangle with both sides of length 1.\\
    Compute $EX$, $EY$, and $EXY$.\\
    \begin{eqnarray*}
      EX & = & \int_0^1 dx \int_0 ^{1-x} 2\,dy\\
      & = & \int_0^1 2x(1-x)\,dx\\
      & = & 2(\frac{1}{2} - \frac{1}{3})\\
      & = & \frac{1}{3}
    \end{eqnarray*}
    $EY = EX = \frac{1}{3}$
    \begin{eqnarray*}
      E(XY) & = & \int_0^{1} dx \int_0^{1-x} xy 2\,dy\\
        & = & \int_0^1 2x\frac{(1-x)^2}{2} \,dx\\
        & = & \int_0^1 (1-x)x^2 \,dx\\
        & = & \frac{1}{3} - \frac{1}{4}\\
        & = & \frac{1}{12}
    \end{eqnarray*}

  \subsection*{Properties of expectation}
    \begin{enumerate}
      \item $E(aX + b) = aE(X) + b$
      \item $E(X_1 + \ldots + X_n) = E(X_1) + \ldots + E(X_n)$
    \end{enumerate}
    \begin{proof}
      Proof of property of two\\
      Take $n = 2$, $E(X+Y) = EX + EY$\\
      In the continuous case,
      $$
        \int\int(x+y)f(x,y)\,dx\,dy = \int\int xf(x,y)\,dx\,dy + \int\int y
        f(x,y)\,dx\,dy
      $$
      Use induction.
    \end{proof}
    \underline{Example}: Assume that an urn contains 10 black, 7 red, and 
      15 white balls. Take 5 balls out.
      \begin{enumerate}
        \item with replacement
        \item without replacement
      \end{enumerate}
      and $X$ is the number of red balls pulled out. Compute $EX$.\\\\
    \underline{"Indicator trick"}: Let $I_i = I\{i\text{th ball is red}\} = 
    \begin{cases}1 & \text{if $i$th ball is red}\\ 0 & \text{otherwise}
    \end{cases}$\\
   Let $X = I_1 + I_2 + I_3 + I_4 + I_5$.\\\\
   In part 1, this is $Binomial(5, \frac{7}{22})$.\\
   You know that $EX = 5 \cdot \frac{7}{22}$, but let's pretend we don't know.\\
   $$
    EI_1 = 1 \cdot P(\text{1st ball is red}) = \frac{7}{22} = EI_2 = \ldots
    = EI_5
   $$
   Therefore, $EX = 5 \cdot \frac{7}{22}$.\\\\
   In part 2, one way you can do it:\\
   $$
    P(X = i) = \frac{\binom{7}{i}\binom{15}{5-i}}{\binom{22}{5}}
   $$
   where $i = 0, 1, \ldots, 5$.\\
   Then, we know that 
   $$
    EX = \sum_{i = 0}^5 i \frac{\binom{7}{i}\binom{15}{5-i}}{\binom{22}{5}}
   $$
   Now, the indicator trick, it turns out to be equal to $5 \cdot 
   \frac{7}{22}$ with the same calculations as the replacement version.\\
    \underline{Example}:
      {\bf "Matching problem"}\\
      $n$ people buys $n$ gifts, which are then assigned at random.\\
      Let $X$ be the number of people who recive their own gift.\\
      What is $EX$?\\
      Let $I_i  I\{\text{Person $i$ receives own gift}\}$\\
      Let $X = I_1 + I_2 + \ldots + I_n$\\
      $EI_i = \frac{1}{n}$\\
      And so, $EX = 1$.\\
      Let $X^2 = I_1 + \ldots + I_n + sum_{i \not= j} I_i I_j$\\\\
    \underline{Example}:
      5 married couples are seated around a round table at random. Let $X$ be
      the number of wives who sit next to their husbands. What is $EX$?\\
      Let $I_i = I\{\text{wife $i$ sits next to her husband}\}$.\\
      Then, $X = I_1 + \ldots + I_5$\\
      $EI_i = \frac{2}{9}$\\
      $EX = \frac{10}{9}$\\\\
    \underline{Example}: {\bf Coupon collector problem}\\
      We have $n$ cards. Sample them with replacement.\\
      Let $N$ be the number of cards you need to sample for complete
      collection.\\
      What is $EN$?\\
      Let $N_1$ be the number of cards to receive of 1st card.\\
      Well, this is trivial. The first card you buy is always the 1st card.\\
      Let $N_2$ be the number of cards to get the second new card.\\
      Let $N_3$ be the number of cards to get the third new card.\\
      $\vdots$\\
      Let $N_n$ be the number of cards to get the $n$th new card.\\
      We know that $N = N_1 + \ldots + N_n$.\\
      $N_i$ is geometric with success, $p = \frac{n-i + 1}{n}$.
      $EN_i = \frac{n}{n-i+1}$\\\\
      $EN = n(1 + \frac{1}{2} + \frac{1}{3} + \ldots + \frac{1}{n})$\\
      Asymptotically, the harmonic sequence above $\approx \log(n)$.
