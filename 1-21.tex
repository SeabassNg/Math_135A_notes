\section*{1/21}
  \underline{Example}: Ten keys with only one that opens the door. 
  Try one by one and discard after each unsuccessful try.\\
  $P(\text{succeed on 5th try}) = \frac{1}{10}$\\
  $P(\text{suceed on or before 5th try}) = \frac{1}{2}$\\\\
  \underline{Example}: Roll a die 12 times.\\
  $P(\text{a number occurs 6 times and two other number occur three 
  times each})$\\
  \begin{enumerate}
    \item Sample Size: $6^{12}$
    \item Choices to pick the first number that occurs 6 times: 
      $\binom{6}{1} = 6$
    \item Choices to pick the next two that occurs 3 times each:
      $\binom{5}{2}$
    \item Pick spots for the number that occurs 6 times:
      $\binom{12}{6}$
    \item Pick spots for one of the numbers that occur 3 times:
      $\binom{6}{3}$
  \end{enumerate}
  Therefore, our probability is $\frac{6\binom{5}{2}\binom{12}{6}
  \binom{6}{3}}{6^{12}}$\\\\
  \underline{Example}: We have 10 pairs of socks in the closet. I will
  pick 8 socks at random.\\
  $P(\text{$i$ complete pair of socks})?$\\
  \begin{enumerate}
    \item Samples size: $\binom{20}{8}$ Pick 8 socks out of 20 socks (10 pairs)
    \item Pick $i$ pairs of socks of the 10: $\binom{10}{i}$
    \item Pick other socks. Make sure that you don't pick pairs again:
      $\binom{10-i}{8-2i}$
    \item Vary the complete pairs: $2^{8-2i}$
  \end{enumerate}
  Therefore, our probability is $\frac{2^{8-2i}\binom{10-i}{8-2i}\binom{10}{i}}
  {\binom{20}{8}}$\\\\
  \underline{Example}: {\bf Poker Hands}\\ 
In the definitions, the word {\it value\/} 
refers to A, K, Q, J, 10, 9, 8, 7, 6, 5, 4, 3, 2.
This sequence also describes the relative rankings 
of cards, with one exception: an Ace may be 
regarded as 1 for the purposes of making a straight:\\ 
%\flushpar 
(a) {\it one pair\/}: two cards of same value plus 3 cards
with different values 
$$
J\sp\,\, J \cl\,\, 9\he\,\, Q\cl\,\, 3\sp
$$
(b) {\it two pairs\/}: two pairs plus another card 
of different value
$$
\,\,J\sp\,\, J \cl \,\,9\he\,\, 9\cl\,\, 3\sp
$$
(c) {\it three of a kind\/}: three cards of the same 
value plus two with different values
$$
J\sp \,\,J \cl\,\, J\he\,\, 9\cl\,\, 3\sp
$$
(d) {\it straight\/}: five cards with consecutive values
$$
5\he\,\, 4\cl\,\, 3\cl\,\, 2\he\,\, A\sp
$$
(e) {\it flush\/}: five cards of the same suit
$$
K\cl\,\, 9\cl\,\, 7\cl\,\, 6\cl\,\, 3\cl
$$
(f) {\it full house\/}: a three of a kind and a pair
$$
J\cl\,\, J\di\,\, J\he\,\, 9\cl\,\, 9\sp
$$
(g) {\it four of a kind\/}: four cards of the same value
$$
J\cl\,\, J\di\,\, J\he\,\, J\cl\,\, 9\sp
$$
(e) {\it straight flush\/}: five cards of the same suit with consecutive values
$$
A\cl\,\, K\cl\,\, Q\cl\,\, J\cl\,\, 10\cl
$$
\vskip0.5cm 
\halign{#\hfill&\hfill#\hfill& \quad\hfill#\hfill\cr
$\text{hand}$ & \hfill $\text{no.~combinations}$  &\hfill $\text{approx.~prob.}$ \hfill\cr
\noalign{\smallskip\hrule\smallskip}
 {\it one pair\/}& $13\cdot {{12}\choose 3} \cdot {4\choose 2} \cdot 4^3$ & 0.422569\cr
{\it two pairs\/}& ${{13}\choose 2} \cdot 11 \cdot {4\choose 2} \cdot {4\choose 2} \cdot 4$ & 0.047539\cr
 {\it three of a kind\/}& $13\cdot {{12}\choose 2} \cdot {4\choose 3} \cdot 4^2$ & 0.021128\cr
{\it straight\/}& $10\cdot 4^5$ & 0.003940\cr
{\it flush\/}& $4\cdot {13\choose 5}$ & 0.001981\cr
{\it full house\/}& $13\cdot 12 \cdot {4\choose 3} \cdot{4\choose 2}$ & 0.001441\cr
 {\it four of a kind\/}& $13\cdot 12\cdot 4$ & 0.000240\cr
 {\it straight flush\/}&$10\cdot 4$ & 0.000015\cr
}
\vskip0.5cm 
%\flushpar
{\it Note.} The probabilities of straight and flush above 
include the possibility of a straight flush. The number 
of all outcomes is ${{52}\choose 5}=2,598,960$. 
Then, for example, for the  {\it three of a kind\/}, the number of 
good outcomes is obtained by choosing the value
for the three cards with the same value, then 
the values of other two cards, then three cards 
from the four of the same chosen value, then 
a card from each of the two remaining chosen values.\\\\ 
  \underline{Assumption}: Every hand is dealt at random\\
  $P(\text{one pair})$?
  \begin{enumerate}
    \item Sample Size: $\binom{52}{5}$ Pick five cards out of 52
    \item Pick a number: 13 choices
    \item Pick the other three numbers that don't make a pair: $\binom{12}{3}$
    \item Pick which pairs of the chosen number: $\binom{4}{2}$
    \item Vary the suits of the other three number: $4^3$
  \end{enumerate}
  Therefore, our probability is $\frac{13\binom{4}{2}\binom{12}{3}}{\binom{52}{5}}$\\\\
  $P(\text{flush})$?
  \begin{enumerate}
    \item Sample Size: $\binom{52}{5}$ Pick five cards out of 52
    \item Pick a suit: 4 choices
    \item Pick five numbers: $\binom{13}{5}$
  \end{enumerate}
  Our probability is $\frac{4\binom{52}{5}}{\binom{13}{5}}$.\\
  This includes straight flushes.\\\\
  $P(\text{straight flush})$?
  \begin{enumerate}
    \item Sample Size: $\binom{52}{5}$ Pick five cards out of 52
    \item Pick a suit: 4
    \item Pick the beginning number: 10. Not cyclic
  \end{enumerate}
  Therefore, our probability is $\frac{4*10}{\binom{52}{5}}$\\\\
  $P(\text{nothing}) = P(\text{all cards with different values}) - 
    P(\text{straight or flush})$?\\
  \begin{enumerate}
    \item Sample Size: $\binom{52}{5}$ Pick five cards out of 52
    \item Pick 5 numbers of 13: $\binom{13}{5}$
    \item Vary the suits: $4^5$
  \end{enumerate}
  Therefore, our probability is $\frac{\binom{13}{5}4^5}{\binom{52}{5}} - 
  (P(\text{straight}) + P(\text{flush}) - P(\text{straight flush}))$\\
  It's a little over half the time.\\\\
  \underline{Example}:{\bf Football players example.} 
  \vskip0.2cm 

  Assume that 20 players, 10 offensive and 10 defensive, are 
  to be distributed at random into 10 rooms, 2 per room. What 
  is the probability that exactly $2i$ rooms are mixed, $i=0,\dots 5$? 

  This is an example when careful thinking about what outcomes 
  should be really pays off. Consider the following 
  model for distributing players into rooms. First 
  arrange them at random onto a row of 20 slots $S1,S2,\dots, S20$. Room 1
  then takes players in slots $S1,S2$, so let's call these two slots $R1$.
  Similarly, room 2 takes players in slots 
  $S3, S4$, so let's call these two slots $R2$, etc. 

  Now it is clear that we only need to keep track of distribution 
  of 10 $d$'s into the 20 slots, corresponding to the positions of 10
  defensive players. Any such distribution constitutes an outcome, 
  and they are equally likely. Their number is ${{20}\choose {10}}$. 

  To get $2i$ (for example, 4) mixed rooms, start by choosing 
  $2i$ (ex., 4) out of 10 rooms which are going to be mixed. 
  There are  ${{10}\choose {2i}}$ of these choices. You also 
  need to make a choice into which slot in each of the 
  $2i$ chosen mixed rooms the $d$ goes, for $2^{2i}$ choices. 

  Once these two choices 
  are made, you still have $10-2i$ (ex., 6) $d$'s to distribute into 
  $5-i$ (ex., 3) rooms, as there are two $d$'s in each of those rooms. 
  For this, you need to choose $5-i$ (ex., 3) rooms out of 
  the remaining $10-2i$ (ex.,  6), for ${{10-2i}\choose {5-i}}$
  choices, and this choice fixes a good outcome. 

  The final answer therefore 
  is 
  $$
  \frac{ {{10}\choose {2i}} 2^{2i} {{10-2i}\choose {5-i}} }{ {{20}\choose {10}} }
  $$

